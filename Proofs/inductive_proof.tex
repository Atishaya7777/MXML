\documentclass{article}

\usepackage{amsmath,amssymb,amsthm,amsfonts,enumerate,enumitem}
\usepackage{hyperref}% http://ctan.org/pkg/hyperref
\usepackage{cleveref}% http://ctan.org/pkg/cleveref

\usepackage{calc} 
\newlength{\casewd}
\setlength{\casewd}{\widthof{\bfseries Case 0.}}%

%%%%%%%%% Environments %%%%%%%%%%%

\newtheorem{theorem}{Theorem}[section]
\newtheorem{lemma}[theorem]{Lemma}
\newtheorem{proposition}[theorem]{Proposition}
\newtheorem{conjecture}[theorem]{Conjecture} 
\newtheorem{cor}[theorem]{Corollary}
\newtheorem{example}[theorem]{Example}
\newtheorem{rem}[theorem]{Remark}
\newtheorem{definition}[theorem]{Definition}
\newtheorem{conj}{Conjecture}[section]
% \newtheorem{procedure}{Procedure}[section]
\newtheorem{question}{Question}[section]
\newtheorem{claim}{Claim}[section]
\newtheorem{quest}{Question for US}[section]
\theoremstyle{definition}

\newlist{caseof}{enumerate}{1}
\setlist[caseof,1]{label = Case \arabic*: , wide=0pt, leftmargin=\dimexpr\casewd + \labelsep, font=\bfseries, topsep=2pt, itemsep=0pt}%

\newlist{subcaseof}{enumerate}{1}
\setlist[subcaseof,1]{label = Case \arabic*: , wide=0pt, leftmargin=\dimexpr\casewd + \labelsep, font=\bfseries, topsep=2pt, itemsep=0pt}%


\begin{document}

\refname{lemma}{Lemma}{Lemmas}
\section{Theorems and Lemmas}

\begin{definition}[Depth]
	The depth starts from 1-index. A single vertex is of depth 1.
\end{definition}

\begin{definition}[Max Independent Set]
	We denote the maximum independent set of a perfect binary tree of depth $d$ by $\mathcal{I}_d$.
\end{definition}

\begin{conjecture}[HK property for a perfect binary tree]
	For any given perfect binary tree $T$, the maximum number of cocliques lie in the leaves. The number of cocliques is denoted by $k$. $\alpha(T)$ denotes the independence number of a tree $T$.
\end{conjecture}

\begin{lemma}\label{independence_num}
	For a given perfect binary tree $T$ with depth $d$ and maximum number of cocliques possible, i.e. $k = \alpha(T)$ we have:
	\begin{align*}
		\alpha(T) = \begin{cases}
			            2^{d - 1} + 2^{d-3} + \ldots + 1 \text{ for odd $d$}  \\
			            2^{d - 1} + 2^{d-3} + \ldots + 2 \text{ for even $d$} \\
		            \end{cases}
	\end{align*}

	Or, in summation notation:
	\begin{align*}
		\alpha(T) = \sum_{i= 0}^{\left\lfloor\frac{d}{2}\right\rfloor} 2^{d - 2i}
	\end{align*}
\end{lemma}

\begin{proof}
	$ $ \\
	We shall proceed by inducing on $d$.\\
	We will have two cases, one for $d$ being odd and another for $d$ being even.
	\begin{caseof}
		\item $d$ is odd \\

		For our base case, consider the trivial case of $d = 1$. Here, $\alpha(T) = 1$. Hence, the base case holds. \\

		Now, say that the statement holds for all $d$. We now have to show that it holds for $d+1$. \\

		We now consider 2 cases:

		\begin{subcaseof}
			\item $Root \in \mathcal{I}_{d + 1}$

			Let the left and right child of the root be $v_0$ and $v_1$ respectively.

			Let $C_1, C_2, C_3, C_4$ be the four perfect binary tree components generated by $T \setminus \{Root, v_0, v_1\}$.

			If the root $\in \mathcal{I}_{d + 1}$, then $v_0 \not\in \mathcal{I}_d$ and $v_1 \not\in \mathcal{I}_d$. This means that the remaining vertices of $\mathcal{I}_d$ is in one of the 4 components.
			Since the current depth from the $Root$ is $d + 1$, then $T \setminus \{Root, v_0, v_1\}$ will have depth of $d + 1 - 2 = d - 1$. Since $d$ is odd, then $d + 1$ is even which implies that $d - 1$ is also even.

			By symmetry, it is enough to consider one of the 4 components' maximum independent set for our calculations. Since the 4 components are disjoint, then we can add their independence numbers together along with the $Root$ and obtain the following:
			\begin{align*}
				\alpha(T_{d + 1}) = 4(\alpha(T_{d - 1})) + 1
			\end{align*}

			Then, from our induction hypothesis, we get that:

			\begin{align*}
				\alpha(T_{d+1}) & = 4(2^{d - 2} + 2^{d - 4} + \ldots + 2) + 1                           \\
				                & = 2^d + 2^{d - 2} + \ldots + 2^3 + 1 \quad  & (i) \label{eq:odd_case}
			\end{align*}

			\item Root $\not\in \mathcal{I}_d$

			If $Root \not\in \mathcal{I}_d$, then the remaining elements of $\mathcal{I}_d$ are from $T \setminus \{Root\}$.

			Let $T' = T\setminus\{Root\}$. Then $T'$ is a forest of 2 disjoint and distinct components. Let $C_1$ and $C_2$ be the 2 components of $T'$. Since $T$ was a perfect binary tree of depth $d + 1$, then $C_1$ and $C_2$ are also perfect binary trees of depth:
			\begin{align*}
				 & = (d + 1) - 1 \\
				 & = d
			\end{align*}

			Since $d$ is odd, and $C_1$ and $C_2$ are disjoint and distrinct perfect binary trees,
			\begin{align*}
				\alpha(T') = \alpha(C_1) + \alpha(C_1)
			\end{align*}

			By symmetry we get that,
			\begin{align*}
				\alpha(T') = 2\alpha(C_1)
			\end{align*}

			Since $Root \not\in \mathcal{I}_d$,
			\begin{align*}
				\alpha(T) = \alpha(T') = 2\alpha(C_1)
			\end{align*}

			Then, by our induction hypothesis,
			\begin{align*}
				\alpha(T) & = 2(2^{d - 1} + 2^{d - 3} + \dots + 1)          \\
				\alpha(T) & = 2^d + 2^{d-2} + \dots + 2\label{eq:even_case}
			\end{align*}
		\end{subcaseof}

		Since $\alpha(T)$ is the maximum independent set,
		\begin{align*}
			\alpha(T) = \max(\ref{eq:odd_case}, \ref{eq:even_case})
		\end{align*}

		From (\ref{eq:odd_case}) and (\ref{eq:even_case}),

		\item $d$ is even
	\end{caseof}
\end{proof}

\begin{lemma}
	Following from the previous lemma ~\ref{independence_num}, we claim that the star whose cardinality is $\alpha(T)$ is unique, i.e there is a unique maximum family of cocliques.
\end{lemma}

\end{document}



