%to present, remove handout from square brackets
\documentclass[10pt,]{beamer}

\usepackage{pgfpages}

\mode<handout>{%
    \pgfpagesuselayout{4 on 1}[a4paper,landscape] 
    %\setbeameroption{show notes}
    \setbeamercolor{note page}{bg=white}
}

\usepackage[]{amsmath}
\usepackage{amsthm,amssymb,amsfonts}
\usepackage{graphicx,xfrac,mathrsfs,subcaption}
\usepackage{tikz,pgfplots}
\usepackage{multicol,multirow}

\usetikzlibrary{positioning,automata}

\tikzset{onslide/.code args={<#1>#2}{%
  \only<#1>{\pgfkeysalso{#2}}
}}


%%%%%%%%%%%%%%%%%%%%%%%%%%%%
%%%%%%%%%%%%%%%%%%%%%%%%%%%%
%%%%% long and short versions
\newcommand{\extra}[1]{}
\newcommand{\short}[1]{#1}

\usetheme{Boadilla}
\makeatother
\setbeamertemplate{footline}
{
	\leavevmode%
	\hbox{%
	\begin{beamercolorbox}[wd=.4\paperwidth,ht=2.25ex,dp=1ex,center]{author in head/foot}%
		\usebeamerfont{author in head/foot}\insertshortauthor
	\end{beamercolorbox}%
	\begin{beamercolorbox}[wd=.6\paperwidth,ht=2.25ex,dp=1ex,center]{title in head/foot}%
		\usebeamerfont{title in head/foot}\insertshorttitle\hspace*{3em}
		\insertframenumber{} / \inserttotalframenumber\hspace*{1ex}
	\end{beamercolorbox}}%
	\vskip0pt%
}
\makeatletter

\setbeamertemplate{navigation symbols}{}
\title[HK-property | Perfect Binary Trees]{Exploring perfect binary trees with relation to the HK-property}
\subtitle{MXML Presentation} %This is where you can specify the conference name or presentation venue
\author[A.M, M.N.S]{Atishaya Maharjan \\ Mahsa N. Shirazi}
\date{\today}

\begin{document}
\parskip = \baselineskip

\begin{frame} %Title page
    \titlepage
\end{frame}

% TODO: Add a citation to Erdos-Ko-Rado Theorems by Chris Godsil
\begin{frame}\frametitle{EKR Theorem}
    \begin{definition}[Intersecting family]
        A family of subsets $\mathcal{F}$ of some set is \textbf{intersecting} if any two members of $\mathcal{F}$ have a non-empty intersection.
    \end{definition}
    \begin{itemize}
        \item The \textbf{Erd\H{o}s-Ko-Rado}  theorem limits the number of sets in an intersecting family.
    \end{itemize}
    \begin{theorem}[EKR Theorem]
        If $\mathcal{F}$ is an intersecting family of $k$-subsets of an $n$-set (cardinality of the set is $n$), then
        \begin{itemize}
            \item $|\mathcal{F}| \leq \binom{n - 1}{k - 1}$
            \item If equality holds, $\mathcal{F}$ consists of the $k$-subsets that contain $i$, for some $i$ in the $n$-set.
        \end{itemize}
    \end{theorem}
\end{frame}

\begin{frame}\frametitle{The Several Variable Case}

    \begin{example}[example title]
        You can add examples
    \end{example}\pause

    \begin{theorem}[Theorem name]
        and theorems
    \end{theorem}\pause %There are fancier ways to make things show up one at a time, but putting in pauses is the easiest

    \begin{definition}[the concept you are defining]
        and definitions, such as \(\mathbb{D} = \left\{z\in \mathbb{C} :\: \left|z\right|<1 \right\}\)
    \end{definition}
\end{frame}


%%%%%%%%%%%%%%%%%%%%%%%%%%%%%%%%%%%%%%%%%%%%%%%%%%%%%%%%%%%%%%%%%%%%%%
%%%%%%%%%%%%%%%%%%%%%%%%%%%%%%%%%%%%%%%%%%%%%%%%%%%%%%%%%%%%%%%%%%%%%%  
%%%%%%%%%%%%%%%%%%%%%%%%%%%%%%%%%%%%%%%%%%%%%%%%%%%%%%%%%%%%%%%%%%%%%%
\begin{frame}\frametitle{Thank You!}
    \framesubtitle{Summary}
    A slideshow usually ends with a summary slide.
\end{frame}

%%%%%%%%%%%%%%%%%%%%%%%%%%%%%%%%%%%%%%%%%%%%%%%%%%%%%%%%%%%%%%%%%%%%%%
%%%%%%%%%%%%%%%%%%%%%%%%%%%%%%%%%%%%%%%%%%%%%%%%%%%%%%%%%%%%%%%%%%%%%%  
%%%%%%%%%%%%%%%%%%%%%%%%%%%%%%%%%%%%%%%%%%%%%%%%%%%%%%%%%%%%%%%%%%%%%%

\end{document}