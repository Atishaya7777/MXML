%% For perfect binary trees
\begin{proof}[Proof for ~\ref{independence_num}]
	$ $ \\
	We shall proceed by inducting on $d$.\\
	We will have two cases, one for $d$ being odd and another for $d$ being even.
	\begin{caseof}
		\item $d$ is odd \\

		For our base case, consider the trivial case of $d = 1$. Here, $\alpha(T) = 1$. Hence, the base case holds. \\

		Now, say that the statement holds for all $d$. We now have to show that it holds for $d+1$. That is, we need to show that it holds for a given perfect binary tree $T$ of depth $d + 1$. Note that if $d$ is odd, then $d + 1$ is even. Hence, we expect that:

		\begin{align*}
			\alpha(T) = 2^d + 2^{d - 2} + \ldots + 2 
		\end{align*}

		Consider these 2 cases:

		\begin{subcaseof}
			\item $r \in \mathcal{I}_{d + 1}$

			Let the left and right child of the root be $v_0$ and $v_1$ respectively.

			Let $C_1, C_2, C_3, C_4$ be the four perfect binary tree components generated by $T \setminus \{Root, v_0, v_1\}$.

			If the root $\in \mathcal{I}_{d + 1}$, then $v_0 \not\in \mathcal{I}_d$ and $v_1 \not\in \mathcal{I}_d$. This means that the remaining vertices of $\mathcal{I}_d$ is in one of the 4 components.
			Since the current depth from the $r$ is $d + 1$, then $T \setminus \{Root, v_0, v_1\}$ will have depth of $d + 1 - 2 = d - 1$. Since $d$ is odd, then $d + 1$ is even which implies that $d - 1$ is also even.

			By symmetry, it is enough to consider one of the 4 components' maximum coclique for our calculations. Since the 4 components are disjoint, then we can add their independence numbers together along with the $r$ and obtain the following:
			\begin{align*}
				\alpha(T_{d + 1}) = 4(\alpha(T_{d - 1})) + 1
			\end{align*}

			Then, from our induction hypothesis, we get that:

			\begin{align}
				\alpha(T_{d+1}) & = 4\underbrace{(2^{d - 2} + 2^{d - 4} + \ldots + 2)}_{\frac{d-1}{2} \text{terms}} + 1                           \nonumber \\
						& = 4 \displaystyle\sum_{i = 1}^{\frac{d-1}{2}}2^{d - 2i} + 1 	
			\label{eq:odd_case}
			\end{align}


		\item Root $\not\in \mathcal{I}_{d + 1}$

			If $r \not\in \mathcal{I}_d$, then the remaining elements of $\mathcal{I}_d$ are from $T \setminus \{Root\}$.

			Let $T' = T\setminus\{Root\}$. Then $T'$ is a forest of 2 disjoint and distinct components. Let $C_1$ and $C_2$ be the 2 components of $T'$. Since $T$ was a perfect binary tree of depth $d + 1$, then $C_1$ and $C_2$ are also perfect binary trees of depth:
			\begin{align*}
				 & = (d + 1) - 1 \\
				 & = d
			\end{align*}

			Since $d$ is odd, and $C_1$ and $C_2$ are disjoint and distrinct perfect binary trees,
			\begin{align*}
				\alpha(T') = \alpha(C_1) + \alpha(C_1)
			\end{align*}

			By symmetry we get that,
			\begin{align*}
				\alpha(T') = 2\alpha(C_1)
			\end{align*}

			Since $r \not\in \mathcal{I}_d$,
			\begin{align*}
				\alpha(T) = \alpha(T') = 2\alpha(C_1)
			\end{align*}

			Then, by our induction hypothesis,
			\begin{align}
				\alpha(T) & = \underbrace{2(2^{d - 1} + 2^{d - 3} + \dots + 1)}_{\left\lceil\frac{d}{2}\right\rceil \text{terms}}          \nonumber \\
					  & = 2\left(\displaystyle\sum_{i = 1}^{\left\lceil\frac{d}{2}\right\rceil}2^{d - (2i - 1)}\right) \nonumber
			\end{align}

			Note that $\left\lceil\frac{d}{2}\right\rceil = \frac{d + 1}{2}$, since $d$ is odd, then,
			\begin{align}
				\alpha(T) & = 2\left(\displaystyle\sum_{i = 1}^{\left\lceil\frac{d}{2}\right\rceil}2^{d - (2i - 1)}\right) \nonumber \\
					  & = 2\left(\displaystyle\sum_{i = 1}^{\frac{d + 1}{2}}2^{d - (2i - 1)}\right)  \label{eq:even_case} 
			\end{align}


		\end{subcaseof}

		Since $\alpha(T)$ is the maximum coclique,
		\begin{align*}
			\alpha(T) = \max(\ref{eq:odd_case}, \ref{eq:even_case})
		\end{align*}

		Remember that our objective is to show that $\alpha(T) = (\ref{eq:even_case})$ as this aligns with our inductive hypothesis.

		From (\ref{eq:even_case}), we can simplify it as the following:

		\begin{align*}
			\alpha(T) & = 2\left(\displaystyle\sum_{i = 1}^{\frac{d + 1}{2}}2^{d - (2i - 1)}\right)  \\
				  & = 2\left(2^{d - 2\left(\frac{d + 1}{2}\right) + 1} + \displaystyle\sum_{i = 1}^{\frac{d - 1}{2}}2^{d - (2i - 1)}\right)\\
				  & = 2\left(2^{d - d - 1 + 1} + \displaystyle\sum_{i = 1}^{\frac{d - 1}{2}}2^{d - (2i - 1)}\right)\\
				  & = 2\left(2^0 + \displaystyle\sum_{i = 1}^{\frac{d - 1}{2}}2^{d - (2i - 1)}\right)\\
				  & = 2\left(1 + \displaystyle\sum_{i = 1}^{\frac{d - 1}{2}}2^{d - (2i - 1)}\right)\\
				  & = 2\left(1 + \displaystyle\sum_{i = 1}^{\frac{d - 1}{2}}2^{d - 2i + 1)}\right)\\
				  & = 2\left(1 + 2\displaystyle\sum_{i = 1}^{\frac{d - 1}{2}}2^{d - 2i)}\right)\\
				  & = 2 + 4\displaystyle\sum_{i = 1}^{\frac{d - 1}{2}}2^{d - 2i}\\
				  & = 4\displaystyle\sum_{i = 1}^{\frac{d - 1}{2}}2^{d - 2i} + 2 \\
				  & > 4\displaystyle\sum_{i = 1}^{\frac{d - 1}{2}}2^{d - 2i} + 1 \\
				  & = (\ref{eq:odd_case})
		\end{align*}

		Which implies that,

		\begin{align*}
			\alpha(T) = \max(\ref{eq:odd_case}, \ref{eq:even_case}) = (\ref{eq:even_case})\; \text{as required.}
		\end{align*}

	\item $d$ is even \\

		We will proceed similarly to the odd case.

		For our base case, consider the case of $d = 2$. Let $v_0$ and $v_1$ be the leaves of the perfect binary tree of depth $2$. We claim that $\mathcal{I}_2 = \{v_0, v_1\}$. This implies that $r \not\in \mathcal{I}_2$. 

		We will prove the claim by using contradiction.	

		For contradiction, say that $r \in \mathcal{I}_2$, then $v_0 \not\in \mathcal{I}_2$ and $v_0 \not\in \mathcal{I}_2$ since both $v_0$ and $v_1$ are neighbours of $r$. This then implies that $\alpha(T) = 1$. However, the set of leaves $\{v_0, v_1\}$ has cardinality $2 > 1 = \alpha(T)$ which is a contradiction.

		Hence, for a perfect binary tree $T$ of depth $2$, $\alpha(T) = 2$. Thus, our base case holds.


		Now, say that the statement holds for all $d$. We now have to show that it holds for $d+1$. That is, we need to show that it holds for a given perfect binary tree $T$ of depth $d + 1$. Note that if $d$ is even, then $d + 1$ is odd. Hence, we expect that:

		\begin{align*}
			\alpha(T) = 2^d + 2^{d - 2} + \ldots + 1
		\end{align*}

		which follows from our previous case.
	\end{caseof}
\end{proof}

