\begin{abstract}
    A perfect binary tree is a full binary tree in which all leaves have the same depth. A set of cocliques of size $k$ ($k$-coclique) in a graph, containing a fixed vertex $v$ is called a star, and is denoted by $\mathcal{I}^n_G(v)$. We study the size of stars for different vertices in a perfect binary tree. This structure is useful in studying the Erd\H{o}s-Ko-Rado theorem. Hurbert and Kumar conjectured that in trees, the largest stars are on the leaves. The conjecture was shown to be false independently by Baber, Borg, and Feghali, Johnson and Thomas. However, in some classes of trees such as caterpillars, the conjecture holds true. In this paper, we study the perfect binary trees through the lens of star centers and seek to answer if the HK-property holds for perfect binary trees. We also aim to expand the definition of the flip function mentioned by Estrugo and Passtine in the context of perfect binary trees. We then use an algorithm by Niskanen and R. J. to generate all cocliques of a perfect binary tree of depth $d$ and compare the number of $k$-cocliques containing a vertex $v$ and a leaf $l$ to see if the HK-property holds for perfect binary trees.
\end{abstract}
